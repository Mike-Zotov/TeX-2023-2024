\documentclass[12pt,twoside]{article}
\usepackage{mathtext}
\usepackage[T2A]{fontenc}
\usepackage[utf8]{inputenc}
\usepackage[english, russian]{babel}
\usepackage[pdftex]{graphicx,rotating}
\usepackage{amssymb}
\usepackage{amsthm}
\usepackage{bm}
\usepackage{color}
\usepackage{latexsym}
\usepackage{titlesec}
\usepackage{amsmath}
\usepackage{amsfonts}
\usepackage{cite}
\usepackage{indentfirst}
\usepackage{enumitem}
\setlist{noitemsep}
\usepackage{cmap}
\usepackage[a4paper, left=25mm, right=20mm, bottom=20mm, top=20mm]{geometry}
\frenchspacing
\pagestyle{plain}
\setlength{\parindent}{0cm}

\begin{document}
\centerline{\bf\large Писменная рецензия команды ЛНМО-7}	
\centerline{\bf\large на задачу \textnumero9 ``Разные средние''}
\centerline{\bf\large команды гимназии \textnumero1 г.~Витебска}
\vspace{6pt}
\subsection*{Резюме по итогам проведённого исследования}
\noindent Заявлены пункты 1, 1.1, 1.2, 2, 2.1, 3. Наша команда считает полностью решёнными только пункт 1; пункты 1.1, 2~--- частично решёнными; пункты 1.2, 2.1 и 3~--- неверно решенными.

\subsection*{Серьёзные ошибки}
\begin{itemize}
\item В пункте 1.1 автор отвечает на один из двух вопросов ``Cколько сыра осталось в магазине после первых $K$ покупателей?'', игнорируя вопрос ``Определите, для каких пар натуральных чисел ($K$, $M$) такое возможно.''
\item В решении пункта 1.2 автор считает что пункт не корректен, однако к нему приводит ответ, без пояснения откуда он взялся. Также наша команда считает, что приведенный автором ответ <<возможно>> не отвечает на вопрос задачи.
\item В пункте 2.1. по аналогии с пунктом 2 получена общая формула, которая неверна. Так как, если в эту формулу подставить данные из пункта 2, то не получается ответ 85~км, который является правильным ответом на пункт 2\\
(см. https://problems.ru/view\_problem\_details\_new.php?id=98513).
\item Из авторского решения пункта 3 следует более сильная оценка про 80\%, которая совсем не следует из решения данного пункта \\
(см. https://problems.ru/view\_problem\_details\_new.php?id=98526), в котором оценка в 75\% и её повысить до 80\% не получится.
\end{itemize}

\subsection*{Недочёты и Опечатки}
\begin{itemize}
\item Высказывание в пунктах 1 и 1.1  ``$\frac{x_1 + x_2}{2} + 10\cdot\frac{x_1 + x_2}{2} = 20$ '',~--- неправда. Мы склонны полагать, что автор случайно поделил на 2 выражение $x_1+x_2$, ибо далее приведены верные рассуждения.
\item В пунктах  1 и 1.1 отсутвуют объяснения в этих и следующих выражениях  $x_1 + x_2 + x_3 = \frac{20\cdot3}{13}\ldots$ в первом пункте и $x_1 + x_2 + x_3 = \frac{100\cdot3}{M+3}\ldots$ во втором пункте.
\item В решении пункта 2 не хватает запятой после этого выражения $60 - xt = \frac{40+xt}{t+\frac{2}{3}}$ из-за чего можно подумать что происходит умножение на $(60-хt)\left(t + \frac{2}{3}\right)$.
\item В ответе пункта 2 ответ написан в $\frac{\text{км}}{\text{ч}}$ хотя просят расстояние.
\end{itemize}

\subsection*{Качественная оценка исследования}
Местами автором была проделана хорошая работа. Однако в пунктах присутствуют как ошибки, так и недочеты, а также в некоторых пунктах отсутствуют объяснения. В целом следует признать, что работа авторами выполнена \textit{средне}.
\end{document}